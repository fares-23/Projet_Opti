\documentclass[11pt, a4paper, oneside, portrait]{report}
\usepackage[utf8]{inputenc}
\usepackage[T2A, T1]{fontenc}
\usepackage[british, french, russian]{babel}
\usepackage[style=ieee]{biblatex}
\usepackage[most]{tcolorbox}
\usepackage{graphicx}
% \usepackage{animate}
\usepackage{xurl}
\usepackage{setspace}
\usepackage{ragged2e}
\usepackage{indentfirst}
\usepackage{mathptmx}
\usepackage{pdfpages}
\usepackage{geometry}
\usepackage{amsmath}
\usepackage{amssymb}
\usepackage{txfonts}
\usepackage{multicol}
\usepackage{fancyhdr}
\usepackage{wrapfig}
\usepackage{array}
\usepackage{float}
\usepackage{alltt}
\usepackage{tabularx}
\usepackage{caption}
\usepackage{fancyvrb}
\usepackage{fvextra}
\usepackage{enumitem}
\usepackage{bigfoot}
\usepackage{hyperref}
\geometry{
    a4paper,
    top=2cm,
    bottom=2cm,
    right=2cm,
    left=2cm
}
\hypersetup{
    colorlinks = true,
    linkcolor = blue,
    urlcolor = blue,
    filecolor = blue,
    citecolor = blue
}
\pagestyle{fancy}
\fancyhf[HC]{\textbf{MU4MEN01 --~Projet d'Optimisation}}\fancyhf[HL]{\thepage}\fancyhf[HR]{\thepage}
\fancyhf[FC]{\thepage}
% \addbibresource{References.bib}
\title{\textbf{MU4MEN01 --~Projet d'Optimisation}}
\author{AKIL Adam, BUCLET Zeca, NOCHÉ Kévin} % Mettre ton nom de famille, Zeca.
\date{\today}


\begin{document}
    \selectlanguage{french}
    \maketitle\thispagestyle{empty} % Flemme de faire plus sophistiqué.
    \newpage\tableofcontents\thispagestyle{empty}

    \newpage\setcounter{page}{1} % Démarrer le rapport à partir de cette page.


    \section*{Introduction}\addcontentsline{toc}{section}{Introduction}
        Dans ce présent document, nous allons étudier un modèle représentant un train, sa consommation, sa batterie et comment optimiser deux choses:

        ---~La capacité de ladite batterie.

        ---~La chute de tension maximale aux bornes du train, sachant que nous ne devons pas atteindre moins de $500$~V aux bornes du train.

        Le modèle auquel nous nous intéressons a été codé en python (le code est joint au document), et prend en compte, de façon non-exhaustive, les déplacements du train, sa vitesse, son accélération, sa puissance consommée et la puissance de la LAC (ligne aérienne de contact).
        Le fichier \texttt{marche.txt}, qui nous a été remis dès le début du projet, indique les déplacements du train en fonction du temps.
        De rapides et simples calculs nous permettent de trouver la vitesse du train, puis son accélération, primordiales pour la suite du projet.

    \section*{La fonction \texttt{Simulation()}}\addcontentsline{toc}{section}{La fonction \texttt{Simulation()}}


    \section*{Monte-Carlo}\addcontentsline{toc}{section}{Monte-Carlo}


    \section*{Algorithme NSGA-II}\addcontentsline{toc}{section}{Algorithme NSGA-II}


    \section*{Conclusion}\addcontentsline{toc}{section}{Conclusion}


\end{document}
